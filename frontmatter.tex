\begin{titlepage}
		\begin{center}
		\vspace*{7em}
		{\scshape\LARGE \name\ 宪法 \par}
		\vspace{1.25em}
		{\scshape\Large Constitution of \name \par}
		\vspace{6.5em}
		{\large\itshape 一个女性主义哲学阅读小组\par}
		\vspace{1em}
		{\large\itshape A Feminist Philosophy Reading Group\par}
		\vfill
		{\large[Date]\par}
		\vspace{1em}
		\end{center}
		
		\newpage % copyright page
		\thispagestyle{empty}
		\setlength{\parindent}{0pt}
		
		\textbf{Credits}\par
		\vspace{1em}
		Article \ref{art:secondary}, Section \ref{sec:strict-scrutiny} uses materials from \url{https://en.wikipedia.org/wiki/Strict_scrutiny}, under the CC BY-SA 3.0 license.\par
		\vspace{1em}
		Article \ref{art:secondary}, Section \ref{sec:communityreview} uses materials from \url{https://en.wikipedia.org/wiki/Wikipedia:Requests_for_adminship}, under the CC BY-SA 3.0 license.
		~\vfill

		Copyright \copyright\ 2018, 2019 \name.\par
		\vspace{1em}
		This work is licensed under a \textit{Creative Commons Attribution-ShareAlike} (CC BY-SA) 4.0 International License. For a summary of the license, see \url{https://creativecommons.org/licenses/by-sa/4.0/}.\par
		\vspace{1em}
		The license may not give you all of the permissions necessary for your intended use. If you would like to request additional permissions, please contact the Coordinators of Aspasia at \href{mailto:aspasiaphilosophy@gmail.com}
		{\ttfamily aspasiaphilosophy@gmail.com}.
	\end{titlepage}

	\newpage
	\vspace*{6em}
	\thispagestyle{empty}
	
	\name\ 的宪法于 [date] 由 \name\ 的 Coordinators 提出,于 [date] 被 Aspasia 社群接受。
	
	\vspace{2em}
	
	\textit{The Constitution of Aspasia was proposed on [date] by the Coordinators of Aspasia, and adopted on [date] by the Aspasia Community.}