\documentclass[12pt]{aspasia-constitution}


\begin{document}
	
	\frontmatter
	
	\begin{titlepage}
		\begin{center}
		\vspace*{7em}
		{\scshape\LARGE \name\ 宪法 \par}
		\vspace{1.25em}
		{\scshape\Large Constitution of \name \par}
		\vspace{6.5em}
		{\large\itshape 一个女性主义哲学阅读小组\par}
		\vspace{1em}
		{\large\itshape A Feminist Philosophy Reading Group\par}
		\vfill
		{\large[Date]\par}
		\vspace{1em}
		\end{center}
		
		\newpage % copyright page
		\thispagestyle{empty}
		\setlength{\parindent}{0pt}
		
		\textbf{Credits}\par
		\vspace{1em}
		Article \ref{art:secondary}, Section \ref{sec:strict-scrutiny} uses materials from \url{https://en.wikipedia.org/wiki/Strict_scrutiny}, under the CC BY-SA 3.0 license.\par
		\vspace{1em}
		Article \ref{art:secondary}, Section \ref{sec:communityreview} uses materials from \url{https://en.wikipedia.org/wiki/Wikipedia:Requests_for_adminship}, under the CC BY-SA 3.0 license.
		~\vfill

		Copyright \copyright\ 2018, 2019 \name.\par
		\vspace{1em}
		This work is licensed under a \textit{Creative Commons Attribution-ShareAlike} (CC BY-SA) 4.0 International License. For a summary of the license, see \url{https://creativecommons.org/licenses/by-sa/4.0/}.\par
		\vspace{1em}
		The license may not give you all of the permissions necessary for your intended use. If you would like to request additional permissions, please contact the Coordinators of Aspasia at \href{mailto:aspasiaphilosophy@gmail.com}
		{\ttfamily aspasiaphilosophy@gmail.com}.
	\end{titlepage}

	\newpage
	\vspace*{6em}
	\thispagestyle{empty}
	
	\name\ 的宪法于 [date] 由 \name\ 的 Coordinators 提出,于 [date] 被 Aspasia 社群接受。
	
	\vspace{2em}
	
	\textit{The Constitution of Aspasia was proposed on [date] by the Coordinators of Aspasia, and adopted on [date] by the Aspasia Community.}
	
	\tableofcontents
	
	\mainmatter
	
	\article{名称、目的与基本原则/Name, Purpose and Basic Principles} \label{art:name}
	
	\section{名称/Name.}
	
	本小组定名为\name。
	
	The Name of this Group shall be \name.
	
	\section{目的/Purpose}\label{sec:purpose} 
	
	本小组的目的为阅读、讨论与传播女性主义哲学与相关学科。
	
	The Purpose of this Group shall be to read, discuss and communicate feminist philosophy and other relevant fields.
	
	\section{基本原则/Basic Principles}\label{sec:principle}
	
	本小组及其成员,在追求本条第\ref{sec:purpose}款所规定的目的时,将遵循下列基本原则。
	
	This Group and its Members, in pursuit of the Purpose stated in Section \ref{sec:purpose} of this Article, shall act in accordance with the following Basic Principles.
	
	\newpage
	
	\begin{enumerate}[leftmargin=1.25cm]
	\item \textit{尊严、地位与权利平等/Equal Dignity, Status and Rights.}\par\vspace{3pt}
	所有成员都将享受平等的尊严、地位与权利。\par
	All Members shall have equal dignity, status and rights.
	\item \textit{非等级制度/No Hierarchy.}\par\vspace{3pt}
	本小组将不设置以等级制度为组织的结构或制度。\par
	This Group shall have no hierarchically organized structures or institutions.
	\item \textit{直接民主/Direct Democracy.}\par\vspace{3pt}
	所有成员将享有有效地参与本小组直接民主决策的平等机会。\par
	All Members shall have equal opportunities to effectively participate in the direct democratic decision making of this Group.
	\item \textit{审议/Deliberation.}\par\vspace{3pt}
	所有成员将享有有效地与其余成员进行审议性对话与辩论的平等机会。\par
	All Members shall have equal opportunities to effectively engage in deliberative discourse and debate with other Members.
	\item \textit{包含/Inclusion.}\par\vspace{3pt}
	本小组将促进对来自系统性代表性不足群体的成员之视角与意见的包含。\par
	This Group shall promote the inclusion of the perspectives and opinions of those Members from systematically underrepresented groups.
	\item \textit{安全且尊重的环境/Safe and Respectful Environment.}\par\vspace{3pt}
	本小组及其成员将致力维持安全且尊重的环境,使任何人都能够自由地表达自我而无需害怕遭到骚扰、攻击或恐吓。\par
	This Group and its Members shall work to maintain a safe and respectful environment where anyone can freely express themselves without fear of being harassed, assaulted or threatened.
	\item \textit{非歧视/No Discrimination.}\par\vspace{3pt}
	本小组及其成员将不会以性、性别、性别认同、性别表达、性取向、肤色、种族、族裔、国籍、年龄、残障、宗教或教育为由歧视任何人。\par
	This Group and its Members shall not discriminate against anyone on the ground of sex, gender, gender identity, gender expression, sexual orientation, color, race, ethnicity, nationality, age, disability, religion or education.
	\item \textit{开放获取/Open Access.}\par\vspace{3pt}
	本小组的知识产出将对公众开放获取。\par
	This Group's intellectual outputs shall be made openly accessible to the general public.
	\end{enumerate}

	\article{成员/Membership} \label{art:membership}
	
	\section{资格/Eligibility} \label{sec:eligibility}
	
	任何个人都可以申请成为本小组成员。
	
	Any individual may apply to become a Member of this Group.
	
	\section{加入/Admission} \label{sec:admission}
	
	个人加入本小组的方式将由社群批准的规则加以规定。
	
	The manner in which individuals are admitted into this Group shall be regulated by rules approved by the Community.
	
	\section{重新加入/Readmission} \label{sec:readmission}
	
	依照本条第\ref{sec:termination}款失去成员身份的个人,可在本条第\ref{sec:admission}款下重新申请成为本小组成员。但依照本条第\ref{sec:expulsion}款被开除的个人,必须在重新申请就其被开除的历史作出解释。
	
	Individuals who had their Membership terminated in accordance with Section \ref{sec:termination} of this Article may reapply under Section \ref{sec:admission} of this Article. However, individuals who were expelled in accordance with Section \ref{sec:expulsion} of this Article must explain their history of expulsion.
	
	\section{信任等级/Trust Levels} \label{sec:trustlevels}
	
	Aspasia 论坛中将设置信任等级以帮助新成员融入社群;以赋予熟悉社群的成员必要的技术能力来维护社群。
	
	Trust levels shall be implemented in the Aspasia Forum to help new Members be involved with the Community; and to give Members familiar with the Community access to necessary technical abilities to maintain the Community.

	\section{终止/Termination} \label{sec:termination}
	
	在下列情况下,成员身份将永久终止:
	
	Under the following circumstances, Membership shall be terminated permanently:
	
	\begin{enumerate}[leftmargin=1.25cm]
		\item 自愿退出本小组;或\par
		voluntarily withdraws from this Group; or
		\item 在过去六个月中没有参加或出席至少25\%的正式活动而缺乏合理解释;或\par
		fails to have participated in or attended at least 25\% of the official events in the last six months without reasonable excuses; or
		\item 依照本条第\ref{sec:expulsion}款被开除;或\par
		is expelled in accordance with Section 6\ref{sec:expulsion} of this Article; or
		\item 去世。\par
		deceases.
	\end{enumerate}

	\section{开除/Expulsion} \label{sec:expulsion}
	
	在不幸发生骚扰、暴力、威胁或其它严重不端行为时,成员可由三分之二多数的表决开除出本小组。除非有一致同意,该表决将以秘密投票为之。
	
	In the unfortunate event of harassment, violence, threat, or any other serious misconduct, a Member may be expelled from this Group by a two-thirds majority vote. The vote shall be by secret ballot except by unanimous consent.
	
	\article{组织/Organization}
	
	\section{非等级制的官员/No Ranked Officers} \label{sec:norankedofficers}
	
	本小组将不设等级制的官员。下述所有职位都将只具更大的责任,而不具更多的权利或更高的地位。
	
	This Group shall have no ranked officers. All posts hereinafter listed shall have only greater responsibilities, and no additional rights or higher statuses.
	
	\section{协调委员会/Coordinating Committee} \label{sec:coordinating}
	
	协调委员会的目的将是使用特定的不能合理赋予全体成员的技术能力。协调委员会将仅依照社群的公意行事。信任等级达到 3 的成员可以通过由第\ref{art:secondary}条第\ref{sec:communityreview}款规定的社群审查过程成为协调委员会的成员。
	
	The purpose of the Coordinating Committee shall be to exercise certain technical abilities whose access cannot be reasonably granted to all Members. The Coordinating Committee shall act only in accordance with the general will of the Community. Members with Trust Level 3 may become members of the Coordinating Committee through the Community Review Process set forth in Article \ref{art:secondary}, Section \ref{sec:communityreview}.
	
	\section{计划委员会/Program Committee} \label{sec:program}
	
	计划委员会的目的是计划本小组的项目、活动和会议。信任等级达到 2 的成员可以自愿加入或退出计划委员会。
	
	The purpose of the Program Committee shall be to plan the programs, events and meetings of this Group. Members with Trust Level 2 may voluntarily join or quit the Program Committee.
	
	\section{会计/Treasurers} \label{sec:treasurers}
	
	在社群认为必要时,一名或多名表现良好的成员可以通过由第\ref{art:secondary}条第\ref{sec:communityreview}款规定的社群审查过程成为会计,负责本小组的财务。
	
	If deemed necessary by the Community, one or more Members may become Treasurer(s) through the Community Review Process set forth in Article \ref{art:secondary}, Section \ref{sec:communityreview}, and be tasked with overseeing this Group's finances.
	
	\article{会议/Meetings} \label{art:meetings}
	
	\section{法定人数/Quorum} \label{sec:quorum}
	
	法定人数将是协调委员会人数乘以 $2$ 或本小组成员数乘以 $0.15$,取二者中较小的那个。任何行政事务必须在满足法定人数时方能开展。
	
	A quorum shall be the number of members of the Coordinating Committee times $2$, or the number of Members of this Group times $0.15$, whichever is smaller. A quorum must be present for any administrative business to be conducted.
	
	\section{不足法定人数/Absence of Quorum} \label{sec:noquorum}
	
	在一次会议不足法定人数时,在该会议计划开始的十五分钟内,会议将会自动延期到下一周的同一日、时间和地点,在延期的会议上的成员将构成法定人数。
	
	In the absence of a quorum at a meeting, within fifteen minutes of that meeting'’s scheduled starting time, the meeting shall automatically be adjourned for the same day,
	time and place in the following week, and those Members present at the adjourned meeting shall constitute a quorum.
	
	\section{常规会议/Regular Meetings} \label{sec:regularmeeting}
	
	协调委员会将至少每月将召开一次常规会议。
	
	The Coordinating Committee shall hold regular meetings at least once a month.
	
	\section{全体会议/Plenary Sessions} \label{sec:plenarysessions}
	
	本小组的全体会议可以由任意三位信任等级达到 2 的成员召集。特别会议的书面通知必须在会议前48小时告知全体成员。
	
	Plenary Sessions of this Group may be called by any combination of three Members with Trust Level 2. Written notice of a plenary session must be communicated to all Members at least 48 hours in advance of the session.
	
	\section{礼仪/Manner} \label{sec:manners}
	
	事务将以有序的、民主的和哲学的礼仪为之,尊重所有人的权利和尊严。在项目、活动与会议中不能遵守此礼仪者可以由主持者点名和谴责。若此人为成员且其表现继续损害到其余成员的权利和尊严和/或本小组的目的和基本原则,可由简单多数表决要求此人离开。并非本小组成员者可在主持的成员之决断下要求其离开。
	
	Business shall be conducted in an orderly, democratic and philosophical manner that respects the rights and dignity of all. An individual who fails to conduct themselves in such a matter at a program, activity or meeting may be named and censured by the person presiding. If the misbehaving individual is a Member and continues to misbehave in a way that is detrimental to the rights and dignity of other Member(s) and/or the Purpose and Basic Principles of this Group, this individual may be asked to leave by a simple majority vote. An individual who is a not a Member of this Group may be asked to leave at the discretion of the person presiding.
	
	\article{会费与捐款/Dues and Donations} \label{art:dues}
	
	\section{会费/Dues} \label{sec:dues}
	
	本小组将不向其成员要求会费。
	
	This Group shall not require dues from its Members.
	
	\section{捐款/Donations} \label{sec:donations}
	
	本小组可以从其成员中接受捐款,以为其活动提供资金。
	
	This Group may accept donations from its Members to finance the activities it engages in.
	
	\article{次级规则/Secondary Rules} \label{art:secondary}
	
	\section{议事规则/Parliamentary Authority} \label{sec:parliamentary}
	
	最新版的《罗伯特议事规则》将在一切罗伯特规则适用,且其与本宪法及本小组所制定或接受的决议、决定、规则、政策与惯例相符合的情况下规制本小组。
	
	The newest edition of \textit{Robert's Rules of Orders} shall govern this Group in all cases to which Robert's Rules are applicable and in which they are consistent with this Constitution and the Resolutions, Decisions, Rules, Policies and Customs made or adopted by this Group.
	
	\section{宪法的至高性/Supremacy of the Constitution} \label{sec:supremacy}
	
	本宪法将具至高性,任何与之相冲突的决议、决定、规则、政策或惯例都将无效。
	
	This Constitution shall be Supreme and any Resolution, Decision, Rule, Policy or Custom in conflict with it shall be invalid.
	
	\section{严格审查/Strict Scrutiny} \label{sec:strict-scrutiny}
	
	在如本小组的安全受到迫近的威胁等非常场合下,协调委员会可采取可能违反第\ref{art:name}条第\ref{sec:principle}款所规定基本原则的权宜措施,但如此的权宜措施必须通过下列的严格审查测试:
	
	Under extraordinary circumstances such as the presence of an imminent threat to the security of this Group, expedient measures that may infringe the Basic Principles set forth in Article \ref{art:name}, Section \ref{sec:principle} may be taken by the Coordinating Committee, but only if such measures pass the following test of Strict Scrutiny:
	
	\begin{enumerate}[leftmargin=1.25cm]
	\item 该权宜措施必须由充分的小组利益证成;且\par
	The expedient measures must be justified by a compelling Group interest; and
	\item  该权宜措施必须严格限于实现该利益;且\par
	The expedient measures must be narrowly tailored to achieve that interest; and
	\item 该权宜措施必须是实现该利益最不具限制性的方式。\par
	The expedient measures must be the least restrictive means for achieving that interest.
	\end{enumerate}

	\section{社群审查过程/Community Review Process}  \label{sec:communityreview}
	
	社群审查过程将以下述步骤为之:
	
	The Community Review Process shall consist of the following steps:
	
	\begin{enumerate}[leftmargin=1.25cm]
		\item \textit{提名/Nomination.} \par\vspace{3pt}
		信任等级达到 3 的成员可以提名自己,或被另一信任等级达到 3 的成员提名。\par
		A Member with Trust Level 3 may nominate themselves or be nominated by another Member with Trust Level 3.
		\item \textit{接受提名/Acceptance of Nomination.} \par\vspace{3pt}
		被提名成员必须在 3 天内表态接受提名,否则社群审查过程将自动终止。\par
		The Nominated Member must announce their acceptance of the Nomination within 3 days. Otherwise, the Community Review Process shall automatically terminate.
		\item \textit{审议/Deliberation.} \par\vspace{3pt}
		在提名被接受后的 7 天内,任何成员都可以审议和辩论提名。表现良好的成员可以在此期间投票,并在投票截止前都可以改变自己的决定。除非一致同意,该投票将不以秘密投票为之,相反应使其对所有成员公开可见。\par
		Within 7 days after the nomination is accepted, any Member may deliberate and debate the nomination. Members in Good Standing may vote in this period, and may change their decisions before the vote ends. The vote shall not be by secrete ballot and instead be made openly accessible to all Members, except by unanimous consent.
		\item \textit{结束/Closing.} \par\vspace{3pt}
		若被提名者获得三分之二多数的赞成票,则将被认为通过社群审查过程。未通过社群审查过程的被提名者在30天内不可以再被提名。\par
		The Community Review Process shall be considered passed if the nominee receives a two-thirds majority vote in favor. Nominees that fail the Community Review Process may not be nominated again within 30 days.
	\end{enumerate}
	
	\section{弹劾/Impeachment} \label{sec:impeachment}
	
	除非自愿辞职,或因第\ref{art:membership}条第\ref{sec:termination}款丧失成员身份,或被弹劾,成员不应失去其通过社群审查过程的职位。弹劾案需由一名信任等级达到 2 的成员提出并由另两名信任等级达到 2 的成员联署。在弹劾案被提出的28天内,协调委员会将调查该案并将其发现向社群报告。协调委员会将在汇报日期的7天以后、14天以内指定 48 小时的表决窗口。弹劾案经由投票成员的三分之二多数的秘密表决通过。
	
	Members shall not lose the posts for which they have passed the Community Review Process, unless they voluntarily resign, or lose their Membership in accordance with Article \ref{art:membership}, Section \ref{sec:termination}, or are impeached. The motion to impeach must be proposed by a Member with Trust Level 2, and seconded by another two Members with Trust Level 2. Within 28 days after the motion to impeach is introduced, the Coordinating Committee shall investigate the case and report its findings to the Community. The Coordinating Committee shall announce a 48-hour voting windows no less than 7 days but no more than 14 days from the date of the report. The motion to impeach shall pass by secret ballot by a two-thirds majority vote of the Members voting.
	
	\section{修正案/Amendments} \label{sec:amendments}
	
	受制于本条第\ref{sec:eternity}款,本宪法的修正案需由一名信任等级达到 2 的成员提出并由另两名信任等级达到 2 的成员联署。若表决的书面通知在7天前告知到全体成员,修正案经由投票成员的三分之二多数的秘密表决生效;或经由全体成员的三分之二多数的秘密表决生效。
	
	Subject to Section \ref{sec:eternity} of this Article, Amendments to this Constitution must be proposed by a Member with Trust Level 2, and seconded by another two Members with Trust Level 2. Amendments shall be ratified by secret ballot by a two-thirds majority vote of the Members voting, if written notice of the vote is communicated to all Members at least 7 days prior to the vote, or by secret ballot by a two-thirds majority vote of the entire Membership.
	
	\section{基本原则的永久性/Eternity of Basic Principles} \label{sec:eternity}
	
	第\ref{art:name}章第\ref{sec:principle}款的修正案将不予考虑。
	
	Amendments to Article \ref{art:name}, Section \ref{sec:principle} shall be inadmissible.
	
	\section{解散/Dissolution} \label{sec:dissolution}
	
	若表决的书面通知在7天前告知到全体成员,本小组经由投票成员的三分之二多数的秘密表决解散;或经由全体成员的三分之二多数的秘密表决解散。
	
	This Group shall be dissolved by secret ballot by a two-thirds vote of the Members voting, if written notice of the vote is communicated to all Members at least 7 days prior to the vote, or by secret ballot by a two-thirds majority vote of the entire Membership.
	
	\backmatter

\end{document}